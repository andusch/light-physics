\documentclass[12pt,a4paper]{article}
\usepackage[utf8]{inputenc}
\usepackage[romanian]{babel}
\usepackage{amsmath}
\usepackage{amsfonts}
\usepackage{amssymb}
\usepackage{graphicx}
\usepackage{tikz}
\usepackage{listings}
\usepackage{xcolor}
\usepackage{hyperref}
\usepackage{geometry}
\usepackage{fancyhdr}
\usepackage{titlesec}
\usepackage{tocloft}

% --- Setări pagină ---
\geometry{margin=2.5cm}
\pagestyle{fancy}
\fancyhf{}
\rhead{Propagarea și Interacțiunea Luminii}
\lhead{Proiect Manim}
\cfoot{\thepage}

% --- Culori pentru cod Python ---
\definecolor{codebg}{HTML}{f5f5f5}
\definecolor{keyword}{HTML}{0070c0}
\definecolor{string}{HTML}{c5060b}
\definecolor{comment}{HTML}{008000}
\definecolor{builtin}{HTML}{aa5d00}

\lstset{
  language=Python,
  backgroundcolor=\color{codebg},
  basicstyle=\ttfamily\small,
  keywordstyle=\color{keyword}\bfseries,
  stringstyle=\color{string},
  commentstyle=\color{comment}\itshape,
  numbers=left,
  numberstyle=\tiny,
  stepnumber=1,
  numbersep=8pt,
  frame=single,
  breaklines=true,
  tabsize=2,
  showstringspaces=false,
  morekeywords={self,def,class,import,from,as,lambda,with,yield,async,await}
}

% --- Titlu ---
\title{%
  \vspace{-1cm}%
  \textbf{Propagarea și Interacțiunea Luminii}\\[0.3em]
  \large Simulări interactive în Python–Manim\\[0.5em]
  \large Documentare Proiect%
}
\author{%
  \large Coordonator: Conf. dr. Ionescu Alin\\[0.5em]
  \large Autor: Student Popescu Andrei\\[0.5em]
  \large Universitatea din București – Facultatea de Fizică\\[0.5em]
  \large An universitar 2025–2026%
}
\date{\today}

\begin{document}

\maketitle
\tableofcontents
\newpage

\section{Introducere}
Proiectul de față ilustrează principalele fenomene ondulatorii și cuantice ale luminii într-un cadru vizual modern, utilizând biblioteca \textbf{Manim} (versiunea 0.18). Materialul este structurat în șase secvențe animate independente, fiecare abordând un concept esențial: propagarea 1D, reflexia, refracția, interferența, modelul fotonic și polarizarea. Secțiunea finală reproduce experimentul double-slit al lui Young într-o interpretare cuantică.

Scopul documentării este triplu:
\begin{itemize}
  \item să ofere profesorilor un instrument didactic complet (cod + explicații);
  \item să servească studenților ca exemplu de implementare a fizicii în Python;
  \item să constituie un punct de plecare pentru extinderi ulterioare (3D, interactivitate, VR).
\end{itemize}

\section{Mediul de lucru}
\begin{itemize}
  \item \textbf{Python} 3.11+ (64-bit)
  \item \textbf{Manim} v0.18.0 (instalat via \texttt{pip install manim})
  \item \textbf{NumPy} $\geq$ 1.24
  \item \textbf{SciPy} (opțional, pentru profiluri de intensitate mai precise)
  \item \textbf{FFmpeg} (pentru generarea clipurilor MP4/GIF)
\end{itemize}

\section{Structura generală a codului}
Fiecare scenă Manim este o clasă care moștenește \texttt{Scene} (2D) sau \texttt{ThreeDScene} (3D). Fluxul tipic este:
\begin{enumerate}
  \item Inițializare: culoare fundal, titlu, formule matematice.
  \item Construirea axelor / obiectelor geometrice.
  \item Animații legate de un \texttt{ValueTracker} (timp, unghi, etc.).
  \item Încheiere cu \texttt{self.wait()} pentru vizualizare.
\end{enumerate}

\section{Propagarea luminii 1D}
\subsection{Modelul fizic}
Unda electromagnetică plane este descrisă de:
\[
E(x,t)=E_0\sin\left[2\pi\left(\frac{t}{T}-\frac{x}{\lambda}\right)\right]
\]
unde $E_0$ este amplitudinea, $T$ perioada, iar $\lambda$ lungimea de undă.

\subsection{Snippet cod – funcția de undă}
\begin{lstlisting}
def wave_function(x, t):
    return E0 * np.sin(2 * np.pi * (t / T - x / wavelength))
\end{lstlisting}

\subsection{Elemente vizuale cheie}
\begin{itemize}
  \item Graficul $E(x)$ redesenat la fiecare cadru cu \texttt{always\_redraw}.
  \item Săgeată roșie care indică viteza de fază $v=\lambda\nu$.
  \item Panou cu parametrii (amplitudine, frecvență, viteză).
\end{itemize}

\subsection{Extinderi propuse}
\begin{itemize}
  \item Adăugarea atenuării $e^{-\alpha x}$.
  \item Impedanță și reflexie la capăt (condiții la limită).
\end{itemize}

\section{Reflexia luminii}
\subsection{Legea reflexiei}
\[
\theta_i=\theta_r
\]
Unghiul de incidență este egal cu unghiul de reflexie, măsurat față de normală.

\subsection{Implementare}
\begin{itemize}
  \item \texttt{Line} pentru suprafață și \texttt{DashedLine} pentru normală.
  \item \texttt{Angle.from\_three\_points} pentru marcarea unghiurilor.
  \item Culori distincte: galben (incident), portocaliu (reflectat).
\end{itemize}

\section{Refracția – Legea lui Snell}
\subsection{Relația fundamentală}
\[
n_1\sin\theta_1=n_2\sin\theta_2
\]
Indicii de refracție $n_1$, $n_2$ sunt ilustrați prin dreptunghiuri colorate (aer albastru, sticlă verde).

\subsection{Calcul numeric}
\begin{lstlisting}
theta2 = np.arcsin((n1 / n2) * np.sin(theta1))
\end{lstlisting}

\subsection{Observații}
\begin{itemize}
  \item Unghiul limită și reflexia totală pot fi adăugate cu un simplu \texttt{if}.
  \item Pentru dispersie, $n(\lambda)$ poate fi implementat cu relația Cauchy.
\end{itemize}

\section{Interferența 3D}
\subsection{Principiu}
Două surse coerente $S_1$ și $S_2$ emit unde sferice; în orice punct $P$ diferența de drum $\Delta r=r_2-r_1$ determină o fază $\Delta\phi=k\Delta r$.

\subsection{Funcție de undă}
\[
\Psi(P,t)=\frac{A}{r_1}\cos(kr_1-\omega t)+\frac{A}{r_2}\cos(kr_2-\omega t)
\]

\subsection{Aspecte tehnice Manim}
\begin{itemize}
  \item \texttt{ThreeDAxes} + \texttt{Surface} parametrică.
  \item \texttt{begin\_ambient\_camera\_rotation} pentru perspectivă dinamică.
  \item \texttt{checkerboard\_colors} evidențiază zonele de intensitate.
\end{itemize}

\section{Modelul fotonic}
\subsection{Energia fotonului}
\[
E=h\nu=\frac{hc}{\lambda}
\]

\subsection{Demo-uri color}
Trei culori (roșu 650 nm, verde 550 nm, albastru 450 nm) sunt „trase” ca fotoni (cercuri mici) de la stânga la dreapta; opacitatea și viteza sunt proporționale cu energia.

\subsection{Discuție conceptuală}
Se poate adăuga o secvență în care fotonii trec printr-o fantă și se obține un pattern de difracție, ilustrând dualismul undă–corpuscular.

\section{Polarizarea – Legea lui Malus}
\subsection{Legea}
\[
I=I_0\cos^2\theta
\]
unde $\theta$ este unghiul dintre planul de polarizare și axa polarizatorului.

\subsection{Control interactiv}
\texttt{ValueTracker} pentru $\theta$; săgețile de ieșire își modifică lungimea și opacitatea în timp real.

\subsection{Material suplimentar}
\begin{itemize}
  \item Polarizare circulară cu fază $\pm\pi/2$.
  \item Activitate practică: două filtre polarizante și un laser de 5 mW.
\end{itemize}

\section{Experimentul double-slit (Young) – versiune cuantică}
\subsection{Context istoric}
Thomas Young (1801) a demonstrat natura ondulatorie a luminii. Interpretarea cuantică modernă: fotonul este într-o superpoziție de a trece prin ambele fante; $|\Psi|^2$ este densitatea de probabilitate de detectare.

\subsection{Etapele animației}
\begin{enumerate}
  \item Sursă cuantică (sferă galbenă + „glow”).
  \item Propagarea particulelor (updater) – ilustrează că traiectoria nu este definită.
  \item Calculul interferenței pe ecran (vectorizat NumPy).
  \item Afișarea pattern-ului (puncte cu opacitate = probabilitate).
\end{enumerate}

\subsection{Rezultate}
\begin{itemize}
  \item Maximele constructive apar când $\Delta r=m\lambda$.
  \item Minimele – când $\Delta r=(m+\tfrac12)\lambda$.
\end{itemize}

\subsection{Cod esențial}
\begin{lstlisting}
psi_total = psi1 + psi2
intensity = np.abs(psi_total)**2
\end{lstlisting}

\section{Concluzii și direcții viitoare}
Proiectul arată că Manim permite traducerea directă a ecuațiilor fizice în obiecte vizuale, menținând rigurozitatea științifică. Următorii pași pot fi:
\begin{itemize}
  \item \textbf{Interactivitate}: integrare cu \texttt{manim-sideview} sau \texttt{Jupyter} pentru slider-e live.
  \item \textbf{Realism optic}: adăugarea coeficienților Fresnel, reflexie totală, dispersie.
  \item \textbf{VR/AR}: export \texttt{.glTF} și vizualizare în Oculus / HoloLens.
  \item \textbf{Optimizare}: utilizarea \texttt{OpenGL} (manim-\texttt{opengl} fork) pentru randare GPU.
\end{itemize}

\section{Referințe}
\begin{enumerate}
  \item Grant Sanderson, \textit{Manim Documentation}, \url{https://docs.manim.community/}, 2025.
  \item Eugene Hecht, \textit{Optics}, 5-th ed., Pearson, 2016.
  \item David J. Griffiths, \textit{Introduction to Quantum Mechanics}, 3-rd ed., Cambridge, 2018.
  \item Richard P. Feynman, \textit{QED: The Strange Theory of Light and Matter}, Princeton, 1985.
\end{enumerate}

\section{Anexa A – Lista completă a fișierelor}
\begin{itemize}
  \item \texttt{wave\_1d.py} – propagare 1D
  \item \texttt{reflection.py} – reflexia pe suprafață plană
  \item \texttt{refraction.py} – legea lui Snell
  \item \texttt{interference3d.py} – interferență 3D
  \item \texttt{photonic.py} – model fotonic
  \item \texttt{polarization.py} – legea lui Malus
  \item \texttt{doubleslit.py} – experiment Young cuantic
\end{itemize}

\section{Anexa B – Comenzi tipice de randare}
\begin{lstlisting}
# Exemplu: generare 1080p, 60 fps, format .mp4
manim -pqh wave_1d.py WavePropagation1D --fps 60

# GIF la calitate medie, 30 fps
manim -pm polarization.py Polarization --format gif
\end{lstlisting}

\vspace{1cm}
\begin{center}
\textit{Document pregătit în \LaTeX{} – sursă disponibilă pe GitHub.}
\end{center}

\end{document}